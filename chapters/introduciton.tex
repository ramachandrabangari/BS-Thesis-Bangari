\section{Motivation}

There has been a great deal of interest in twisted Moiré superlattice since the discovery of correlated insulating states and superconductivity in magic angle twisted bilayer graphene (twBLG). \cite{Cao2018} \cite{Cao2018_2} This gave rise to a new field called “twistronics”. \cite{Ribeiro-Palau690} This is a new platform hosting strong electron correlations, providing an alternative for understanding unconventional superconductivity. Transport measurements show us the different phases exhibited by twBLG due to strong correlation, including ferromagnetism and quantum anomalous Hall effect. \cite{Serlin900,Sharpe605} Tunneling measurements shed some light on the band structure of the material involved, and hence can be a great probe in digging deep into this feature-rich heterostructure. \cite{Mishchenko2014} They also help in finding the lifetime of the electrons within a two-dimensional electron gas (2DEG), which gives information about the scattering processes involved. \cite{Ritchie} From the width of the tunneling resonance, the lifetime of the electrons within a 2DEG can be calculated, which tells us about the density dependence of the electron-impurity scattering and the temperature dependence of the electron-electron scattering. Phonon-assisted tunneling is also possible where phonons can provide the momentum needed to overcome the Fermi surface mismatch and allow tunneling. \cite{Chandni}

\section{Overview}

An electron in a crystal is subjected to a periodic potential, obeying dispersion from band structure in reciprocal space. In superlattices, the secondary larger periodic potential breaks the reciprocal space by introducing miniband structures. Fabricating an artificial superlattice is difficult in a common crystal. However, in a two-dimensional (2D) system like graphene, it is easy to create superlattice just by twisting two layers, \cite{Hunt1427,Yankowitz2012,Dean2013,Ponomarenko2013,Yang2013,Schmidt2014} giving rise to Moiré superlattice with a long wavelength, shown by developments in 2D van der Waals heterostructures. \cite{Geim2013}

For example, the long-periodical potential in graphene/hexagonal boron nitride (hBN) Moiré superlattice reshapes
the linear band structure by producing superlattice minibands, leading to the observations of fractal Hofstadter butterfly spectrum and satellite secondary insulating states. \cite{Iwasaki2020}

Unlike graphene/hBN heterostructure, twisted bilayer graphene is able to tune not only the size of the Moiré band but also the electronic coupling between the two layers. This changes a single particle non-interacting picture to a many-body interacting picture. It was predicted that the electronic Fermi velocity almost vanishes around zero Fermi energy at a discrete set of angles. \cite{Bistritzer12233} At these “magic angles”, flat bands are formed that lead to correlation effect dominated electronic behaviour when the bandwidth is smaller than the Coulomb repulsion energy. We will discuss more about the theory in section \ref{section:a}

Despite the prediction of the magic angle in 2011, such magic-angle samples were not experimentally realised until 2018. The first and biggest challenge is the accurate control of the twist angle while fabricating the heterostructure. The breakthrough was the development of the ‘tear and stack’ technique, \cite{Kim16} which involves tearing a single flake of graphene into two by an hBN flake and picking up one of them, then rotating the bottom graphene flake by an angle and picking it up. We will discuss more about the methods used in fabricating twBLG in section \ref{section:c}.

Another problem is the dynamic instability of a small angle in the twisted bilayer graphene after the transfer. Since the most energetically stable stacking is Bernal or AB stacking, i.e., zero twist angle, magic angle samples with a twist of $1.1^o$ \cite{Bistritzer12233} will tend to decrease to zero upon application of strain, heat, etc. \cite{WangThermal,Woods2016}

Renormalisation of Fermi velocity, \cite{Cao2016,Yin2015,Andrei2011} insulating states at the superlattice band edges \cite{Cao2016} and satellite Landau fans were observed \cite{Schmidt2014,Cao2016} even before the magic angle samples were realised. They indicated the possiblity of exotic phenomena near magic angle, even though they didn't show any. Eventually, Pablo Jarillo-Herrero's group at MIT could obtain magic angle samples and observe the correlated insulating states and superconductivity. \cite{Cao2018, Cao2018_2} We will discuss the signatures of twBLG in section \ref{section:b}.

Some works on tunneling measurements were done parallelly on graphene-insulator-graphene and bilayer graphene-insulator-bilayer graphene. \cite{Mishchenko2014, Tutuc, Tutuc17, Tutuc18} We will discuss them in section \ref{section:t}, along with tunneling in graphene-insulator-metal heterostructure. \cite{chaves2013model}

We will discuss the protocols for measurement and results from the twBLG device we measured in section \ref{section:m}. The signatures suggest that our device has a low twist angle between the graphene layers.
