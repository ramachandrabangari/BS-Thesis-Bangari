% Packages you intend to use
% ..

% For example, if you want to render 
% the document in a different font you can
% use something like: 

% \usepackage{gentium}
\usepackage{float}
%\usepackage{subfig}
\usepackage{graphicx}
\restylefloat{table}
\usepackage{url}
\usepackage{listings}
\usepackage[table,xcdraw]{xcolor}
\usepackage{xcolor}
\usepackage{bm}
\usepackage[backend=biber,sorting=none]{biblatex}
\usepackage{caption}
\usepackage{enumitem}
\usepackage{amsmath,amssymb}
\usepackage{physics}
\definecolor{codegreen}{rgb}{0,0.6,0}
\definecolor{codegray}{rgb}{0.5,0.5,0.5}
\definecolor{codepurple}{rgb}{0.58,0,0.82}
\definecolor{backcolour}{rgb}{0.95,0.95,0.92}

\lstdefinestyle{mystyle}{
    backgroundcolor=\color{backcolour},   
    commentstyle=\color{codegreen},
    keywordstyle=\color{magenta},
    numberstyle=\tiny\color{codegray},
    stringstyle=\color{codepurple},
    basicstyle=\ttfamily\footnotesize,
    breakatwhitespace=false,         
    breaklines=true,                 
    captionpos=b,                    
    keepspaces=true,                 
    numbers=left,                    
    numbersep=5pt,                  
    showspaces=false,                
    showstringspaces=false,
    showtabs=false,                  
    tabsize=2
}

\lstset{style=mystyle}