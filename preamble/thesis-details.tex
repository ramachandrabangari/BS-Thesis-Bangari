% Put the correct details in here
\author{S Ramachandra Bangari}
\thesisDegree{Bachelor of Science (Research)}
\thesisSubmissionDate{June 2021}

% If your thesis title spans over three lines, prepend the command with \Large!
\title{\bfseries Exploring Electronic Properties Of Twisted Bilayer Graphene}

% Supervisor details
\thesisSupervisor{\\ Dr. Chandni U, \\ Department of Instrumentation and Applied Physics, \\ Indian Institute of Science}